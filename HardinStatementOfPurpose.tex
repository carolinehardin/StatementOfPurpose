\documentclass[12pt]{letter}

\usepackage{newcent} % Default font is the New Century Schoolbook PostScript font

\usepackage[protrusion=true,expansion=true]{microtype} % Better typography

\newcommand{\ignore}[2]{\hspace{0in}#2} %for easy inline commenting that doesn't create wonky spacing

% Margins
\topmargin=-1in % Moves the top of the document 1 inch above the default
\textheight=8.5in % Total height of the text on the page before text goes on to the next page, this can be increased in a longer letter
\oddsidemargin=-10pt % Position of the left margin, can be negative or positive if you want more or less room
\textwidth=6.5in % Total width of the text, increase this if the left margin was decreased and vice-versa
\setlength\parindent{24pt}
%----------------------------------------------------------------------------------------

\begin{document}
\begin{letter}
%----------------------------------------------------------------------------------------
%	YOUR NAME & ADDRESS SECTION
%----------------------------------------------------------------------------------------

\noindent 
\large\textbf{Caroline D. Hardin\\
Statement of Purpose\\
For School of Education\\
Curriculum \& Instruction\\
Digital Media
}
\vfill


\signature{Caroline D. Hardin} % Your name for the signature at the bottom

%----------------------------------------------------------------------------------------
%	LETTER CONTENT SECTION
%----------------------------------------------------------------------------------------

With computers being one of the most significant tools humanity has created to augment its abilities, the importance of encouraging all people to move from simply using technology to creating it cannot be overstated. Unfortunately, racial and gender disparities in both access to technology and in its development are well-documented---and this lack of diversity diminishes us all. The question, then, is ``how do we fix it''---but answers are often anecdotal, and lack the heft of data or even repeatability. Plucky Kickstarters, over-hyped industry-funded websites, and non-profit after school programs can all be parts of the solution, but what is most urgently needed now is more researchers searching for truth and creating solutions that can reach every child. After all, it is researchers in academic settings which have initiated many of the projects with the most promise, such as Studio K, Computer Clubhouse, Lilypad Arduino, Scratch, and ARIS. \ignore{I just don't think you can pretend not to know Scratch with a straight face.} I wish to join the ranks of those creating new technologies and media in the service of learning and human growth, with the academic and research background to ground and support my creations.

I am fortunate enough to have worked on several digital media projects in pursuit of my Master's degree. I found it very rewarding to design games informed by learning theory, as well as apps and constructionism classroom activities; I have also enjoyed many opportunities to write or speak about the work we do in Curriculum and Instruction. \ignore{why is this (C\&I) capitalized? ---BFO} Event organizing and community outreach, whether for a hackathon or an all-girls Scratch club, has been particularly exciting, driving me to volunteer at a number of events. Insights from these events guides my ongoing thesis work on an open-source social pinboard site for use by girls in STEM outreach programs.

Beyond my current thesis work, I have several other sources of inspiration and perspective on this space. I spent three years with the Peace Corps in Ghana, teaching computer science to INSERT HEARTWARMING DESCRIPTION. My current job at the UW Academic Technology department has given me an additional perspective on how institutions and faculty interact with instructional technology; moderating the MOOC forums, designing a badging system, writing code for SIFTR, and running ARIS workshops have all given professional focus to my classroom work.

I wish to pursue my PhD with Matthew Berland in his Complex Play Lab so that I may continue this work on research-based computer science education for underserved groups using games and open source software. In particular, my areas of interest are wearable electronics, hackerspaces, and augmented reality. In line with my history of service-based learning, the Wisconsin Idea fundamentally inspires my application to the UW-Madison School of Education PhD program as the place where research-backed theory has a real impact on everyday people. With my PhD, I plan to create new educational opportunities in technology education, not just for my own students, but for students around the world; my hope is that their diversity of interests and perspectives can contribute to our shared culture, our shared goals, and the ongoing dialogue about how to shape our shared civilization moving forward.

\closing{Respectfully yours,}

\end{letter}

\end{document}
