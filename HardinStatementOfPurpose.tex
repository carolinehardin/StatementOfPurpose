\documentclass[12pt]{letter}

\usepackage{newcent} % Default font is the New Century Schoolbook PostScript font

\usepackage[protrusion=true,expansion=true]{microtype} % Better typography

\newcommand{\ignore}[2]{\hspace{0in}#2} %for easy inline commenting that doesn't create wonky spacing

% Margins
\topmargin=-1in % Moves the top of the document 1 inch above the default
\textheight=8.5in % Total height of the text on the page before text goes on to the next page, this can be increased in a longer letter
\oddsidemargin=-10pt % Position of the left margin, can be negative or positive if you want more or less room
\textwidth=6.5in % Total width of the text, increase this if the left margin was decreased and vice-versa
\setlength\parindent{24pt}
%----------------------------------------------------------------------------------------

\begin{document}
\begin{letter}
%----------------------------------------------------------------------------------------
%	YOUR NAME & ADDRESS SECTION
%----------------------------------------------------------------------------------------

\noindent 
\large\textbf{Caroline D. Hardin\\
Statement of Purpose\\
For School of Education\\
Curriculum \& Instruction\\
Digital Media
}
\vfill


\signature{Caroline D. Hardin} % Your name for the signature at the bottom

%----------------------------------------------------------------------------------------
%	LETTER CONTENT SECTION
%----------------------------------------------------------------------------------------

With computers being the most powerful tools humans have ever invented, the importance of encouraging all people to move beyond being consumers of technology to become it's creators cannot be overstated . Yet the gender, racial and class disparities in computer science and game development are well documented. It is encouraging that in the last few years this problem is increasingly acknowledged. But when I hear the question 'so how do we fix it?', frequently the answers are anecdotal and lack the heft of data. Plucky kickstarters, glossily hyped industry funded websites, and non-profit after-school programs are important parts of the solution, but what is most urgently needed now is more researchers searching for truth and creating solutions that can reach every child. After all, it is researchers in academic settings which have initated the projects with the most promise, such as Studio K, Computer Clubhouse, Lilypad Arduino, and ARIS. \ignore{is it bad I mention 3 MIT Projects to one UW Madison? Ok I changed scratch to StudioK} It is this tradition of \"new technologies and media in the service of learning and human growth\" I wish to join.

I have been fortunate to already have worked on a wide variety of digital media projects in pursuit of my Master's degree. I found it very rewarding to design theoretically informed games, apps and constructionism classroom activities, and have enjoyed many opportunities to write or speak about the work we do in Curriculum and Instruction. Event organizing and community outreach, whether for a hackathon or an all-girls Scratch club, has been particularily exciting, driving me to volunteer at a number of events. It is the insights from these which is guiding my thesis work on an open-source social pinboard site for use by girls-in-STEM outreach programs.

My job at the UW Academic Technology department has also given me invaluable insight around how institutions and faculty interact with instructional technology. Whether \ignore{am I using 'whether' too much?} moderating the MOOC forums, designing a badging system, writing code for SIFTR or running ARIS workshops, these on-the-ground applications have given professional focus to my classroom work.

I wish to pursue my PhD with Matthew Berland and his Complex Play Lab so that I may continue this work on research based computer science education for underserved groups using games and open source software.\ignore{am I repeating myself?} In particular, my areas of interest are wearable electronics, hackerspaces, and augmented reality. In line with my history of service based learning, the Wisconsin Idea fundamentally inspires my application to the UW-Madison School of Education PhD program as the place where research backed theory has real impact on everyday people. \ignore{something optimistic and concludy needed here}

\closing{Respectfully yours,}

\end{letter}

\end{document}
