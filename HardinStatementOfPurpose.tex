\documentclass[12pt]{letter}

\usepackage{newcent} % Default font is the New Century Schoolbook PostScript font

\usepackage[protrusion=true,expansion=true]{microtype} % Better typography

% Margins
\topmargin=-1in % Moves the top of the document 1 inch above the default
\textheight=8.5in % Total height of the text on the page before text goes on to the next page, this can be increased in a longer letter
\oddsidemargin=-10pt % Position of the left margin, can be negative or positive if you want more or less room
\textwidth=6.5in % Total width of the text, increase this if the left margin was decreased and vice-versa
\setlength\parindent{24pt}
%----------------------------------------------------------------------------------------

\begin{document}
\begin{letter}
%----------------------------------------------------------------------------------------
%	YOUR NAME & ADDRESS SECTION
%----------------------------------------------------------------------------------------

\noindent 
\large\textbf{Caroline D. Hardin\\
Statement of Purpose\\
For School of Education\\
Curriculum \& Instruction\\
Digital Media
}
\vfill


\signature{Caroline D. Hardin} % Your name for the signature at the bottom

%----------------------------------------------------------------------------------------
%	LETTER CONTENT SECTION
%----------------------------------------------------------------------------------------

The gender, racial and class disparities in computer science and game development are well documented. With computers being the most powerful tools humans have ever invented, the importance of not leaving large segments of our population as consumers instead of creators cannot be overstated. It is encouraging that in the last few years this problem has begun to get some of the attention it deserves. Still, I hear the question 'how do we fix it' primarily answered by anecdote. While plucky kickstarters, glossily hyped industry funded websites, and non-profit after school programs are welcome, what is most urgently needed is more researchers searching for truth and creating solutions that can reach every child. After all, the projects with the most promise: Scratch, Computer Clubhouse, Lilypad arduino, and ARIS, have all emerged from researchers in academic settings.

I have been fortunate to work on a wide variety of digital media projects in pursuit of my Master's degree. I found designing theoretically informed games and apps to be very rewarding, and enjoyed many opportunities to write or speak about the work we do in Digital Media. I found event organizing and community outreach, whether a hackathon or an all-girls Scratch club, to be particularily exciting, driving me to volunteer at a number of events. It is the insights from these which is guiding my thesis work on an open-source social pinboard site for use by girls-in-STEM outreach programs.

My job at the UW Academic Technology department has also given me invaluable insight around how institutions and faculty interact with instructional technology. Whether moderating the MOOC forums, designing a badging system, writing code for SIFTR or running ARIS workshops, this work has 

I wish to pursue my PhD with Matthew Berland and his Complex Play Lab so that I may continue to work on computer science education for underserved groups using games and open source software. 

%Since starting my Master's degree program, I've been thrilled to participate in many opportunities to advance the field of computational literacy and computer science education for women and other underserved groups. I’ve created games for language learning, designed apps for studying, ran experiments using virtual reality, designed classroom activities for metacognitive constructionism, created and field tested a wearable electronics curriculum, spoken at a conference about non-traditional education paths, created an augmented reality game for exploring UW campus resources, organized a women’s hackathon, updated the Wisconsin Science Festival CS activities, organized a class tour of a hackerspace, led an all girls scratch club at the Madison Children’s Museum, and volunteered in leadership positions at a number of conferences. My thesis project is developing an open-source social pinboard site for use by girls-in-STEM outreach programs.

%During this time, I worked for the UW’s Academic Technology. My work experience here includes moderating the forums for the first UW Mooc, instructional design support for a blended course, designing a badging system, supporting faculty in using academic technology, designing an augmented reality games with ARIS, creating case scenario/critical readers, writing javascript code for an augmented reality social tagging app (SIFTR).

%My area of interest remains how to lower the barriers for entry into computer science and computational literacy for women and other underserved groups, with a particular focus on open source academic technology and games. I came to recognize that one of the best ways to advance the state of the field was to continue to contribute to serious scholarship, research and outreach. Therefore, I am applying to earn my PhD with Matthew Berland and his Complex Play Lab in the Digital Media Program at the University of Wisconsin-Madison’s Curriculum and Instruction department. 

%My statement of purpose for my Master’s application says that ``After I`ve earned my Master's degree, my goal is to partner with nonprofits such as My Class Cares or Room to Read to develop high-quality, open-source, community-collaborative IT educational resources and games.'' I still aspire to contribute to these worthy causes, but see that a better way to leverage my energies is by using scholarship to create resources, and research to verify the effectiveness of these resources. I take inspiration from a variety of important projects in the area of CS education, such as Scratch, ARIS, and Computer Clubhouse, all of which emerged from an academic setting. Being in an PhD program also creates ample opportunities for me to advocate for improvements in the field through writing and publishing. 

%Especially in this era of unprecedented challenges for higher education, we need new leaders with a background in digital media, instructional technology and computational literacy. I deeply believe in the Wisconsin Idea and I want to be at the UW because of it’s dedication to making an impact beyond the classroom. I’m excited by the challenges of pursuing a PhD and am ready to take them on!

\closing{Sincerely yours,}




%----------------------------------------------------------------------------------------
%	ESSAY BODY
%----------------------------------------------------------------------------------------





\end{letter}

\end{document}
